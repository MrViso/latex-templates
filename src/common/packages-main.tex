% Load the language class
\RequirePackage{babel}

% This package provides expandable checks for the current language based on macro \languagename or hyphenation patterns.
% The package is part of the oberdiek bundle.
\RequirePackage{iflang}

% The xspace package provides a single command that looks at what comes after it in the command stream, and decides whether to insert a space to replace one "eaten" by the TeX command decoder. The decision is based on what came after any space, not on whether there was a space (which is unknowable): so if the next thing proves to be punctuation, the chances are there was no space, but if it's a letter, there's probably a need for space. This technique is not perfect, but works in a large proportion of cases.
% The package is part of the latex-tools bundle in the LaTeX required distribution. 
\RequirePackage{xspace}

% Fontspec is a package for X3LaTeX and LuaLaTeX. It provides an automatic and unified interface to feature-rich AAT and OpenType fonts through the NFSS in LaTeX running on X3TeX or LuaTeX engines.
% The package requires the l3kernel and xparse bundles from the LaTeX 3 development team. 
\RequirePackage{fontspec}

% The package supports the Text Companion fonts, which provide many text symbols (such as baht, bullet, copyright, musicalnote, onequarter, section, and yen), in the TS1 encoding.
% Note that the package has been adopted as part of the LaTeX distribution; the reference here is to the original package, which is now little used (if at all). 
\RequirePackage{textcomp}

% Tune the output format of dates according to language. This package provides ten output formats of the commands \today, \printdate, \printdate-TeX, and \daterange (partly language dependent). Formats available are: ISO (yyyy-mm-dd), numeric (e.g.,dd.\,mm.~yyyy), short (e.g.,dd.\,mm.\,yy), TeX (yyyy/mm/dd), original (e.g., dd. mmm yyyy), short original (e.g., dd. mmm yy), as well as numerical formats with Roman numerals for the month. The commands \printdate and \printdate-TeX print any date. The command \daterange prints a date range and leaves out unnecessary year or month entries. This package supports German (old and new rules), Austrian, US English, British English, French, Danish, Swedish, and Norwegian.
\RequirePackage{isodate}

% Latin modern font package
\RequirePackage{lmodern}

% The package provides the principal packages in the AMS-LaTeX distribution. It adapts for use in LaTeX most of the mathematical features found in AMS-TeX; it is highly recommended as an adjunct to serious mathematical typesetting in LaTeX. 
\RequirePackage{amsmath}

% Additional math symbolys for a list see http://milde.users.sourceforge.net/LUCR/Math/mathpackages/amssymb-symbols.pdf
\RequirePackage{amssymb}

% Additional math symbols
\RequirePackage{dsfont}

% A package which allows the user to set tensor-style super- and subscripts with offsets between successive indices. It supports the typesetting of tensors with mixed upper and lower indices with spacing, also typset preposed indices. This is a complete revision and extension of the original ‘tensor’ package by Mike Piff.
\RequirePackage{tensor}

% Adds infix expressions to perform arithmetic on the arguments of the LaTeX commands \setcounter, \addtocounter, \setlength, and \addtolength. Since many packages start their arithmetic by storing an argument in a register, the package has wide applicability.
% This package is part of the latex-tools bundle in the LaTeX required distribution. 
\RequirePackage{calc}

% Improves the interface for defining floating objects such as figures and tables. Introduces the boxed float, the ruled float and the plaintop float. You can define your own floats and improve the behaviour of the old ones. The package also provides the H float modifier option of the obsolete here package. You can select this as automatic default with \floatplacement{figure}{H}.
\RequirePackage{float}

% Provides support for setting the spacing between lines in a document. Package options include singlespacing, onehalfspacing, and doublespacing. Alternatively the spacing can be changed as required with the \singlespacing, \onehalfspacing, and \doublespacing commands. Other size spacings also available. 
\RequirePackage{setspace}

% The package provides an easy and flexible user interface to customize page layout, implementing auto-centering and auto-balancing mechanisms so that the users have only to give the least description for the page layout. For example, if you want to set each margin 2cm without header space, what you need is just \usepackage[margin=2cm,nohead]{geometry}.
% The package knows about all the standard paper sizes, so that the user need not know what the nominal 'real' dimensions of the paper are, just its standard name (such as a4, letter, etc.).
% An important feature is the package's ability to communicate the paper size it's set up to the output (whether via DVI \specials or via direct interaction with pdf(La)TeX). 
\RequirePackage{geometry}

% This package was developed as a general solution to the problem of including graphics in LaTeX 2.09; as such there are obsolete copies to be found on the web (though no longer on the archive). These old versions should not be used with current LaTeX.
% The current 'preferred' solution is the LaTeX graphicx package, but the graphics bundle does contain a version of epsfig for use with current LaTeX. 
\RequirePackage{epsfig}

% The package builds upon the graphics package, providing a key-value interface for optional arguments to the \includegraphics command. This interface provides facilities that go far beyond what the graphics package offers on its own.
% For extended documentation, see epslatex.
% The package is part of the latex-graphics bundle, which is one of the collections in the LaTeX 'required' set of packages. 
\RequirePackage{graphicx}
\graphicspath{{figures/}{images/}}

%
\RequirePackage{tikz}
% PGFPlots draws high-quality function plots in normal or logarithmic scaling with a user-friendly interface directly in TeX. The user supplies axis labels, legend entries and the plot coordinates for one or more plots and PGFPlots applies axis scaling, computes any logarithms and axis ticks and draws the plots, supporting line plots, scatter plots, piecewise constant plots, bar plots, area plots, mesh-- and surface plots and some more.
% Pgfplots is based on PGF/TikZ (PGF); it runs equally for LaTeX/TeX/ConTeXt. 
\RequirePackage{pgfplots}
\RequirePackage{tikzscale}
\pgfplotsset{compat=newest}
\ifDocumenttikz
  % The package provides straightforward ways to define three-dimensional coordinate frames through which to plot in TikZ. The user can specify the orientation of the main coordinate frame, and use standard TikZ commands and coordinates to render their tikzfigure. A secondary coordinate frame is provided to allow rotations and translations with respect to the main coordinate frame. In addition, the package can also handle plotting user-specified functions in spherical polar coordinates, where both the radius and fill color can be expressed as parametric functions of polar angles. 
  \RequirePackage{tikz-3dplot}
\fi


% The package has a lot of flexibility, including an option for specifying an entry at the "natural" width of its text.
% The package is distributed with the bigdelim and bigstrut packages, which can be used to advantage with \multirow cells.
\RequirePackage{multirow}

% The package starts from the basic facilities of the color package, and provides easy driver-independent access to several kinds of color tints, shades, tones, and mixes of arbitrary colors. It allows a user to select a document-wide target color model and offers complete tools for conversion between eight color models. Additionally, there is a command for alternating row colors plus repeated non-aligned material (like horizontal lines) in tables. Colors can be mixed like \color{red!30!green!40!blue}.
\RequirePackage{xcolor}
% Defines the 16 colors from Ethan Schoonover's Solarized palette
\RequirePackage{xcolor-solarized}

% The bm package defines a command \bm which makes its argument bold. The argument may be any maths object from a single symbol to an expression. This is closely related to the specification of the \boldsymbol command in AMS-LaTeX, but \bm is rather more careful in the way it does things.
\RequirePackage{bm}

% The package is intended for setting rich text into titling capitals (in which the first character of words are capitalized). It automatically accounts for diacritical marks (like umlauts), national symbols (like "ae"), punctuation, and font changing commands that alter the appearance or size of the text. It allows a list of predesignated words to be protected as lower-cased, and also allows for titling exceptions of various sorts.
\RequirePackage{titlecaps}
\Addlcwords{the,a,an,and,but,for,or,nor,to,with,without}

% The titling package provides control over the typesetting of the \maketitle command and \thanks commands, and makes the \title, \author and \date information permanently available. Multiple titles are allowed in a single document. New titling elements can be added and a titlepage title can be centered on a physical page. 
\RequirePackage{titling}

% Intermix single and multiple columns. Multicol defines a multicols environment which typesets text in multiple columns (up to a maximum of 10), and (by default) balances the end of each column at the end of the environment. The package enables you to switch between any (permitted) number of columns at will: there is no imposed "clear page" operation, as there is in unadorned LaTeX at a switch between \onecolumn and \twocolumn sections. The multicolumn environment can also be used inside a box, thus allowing multicolumned insets in text. Multicol patches the output routine, and may clash with other packages that do the same (e.g., longtable); furthermore, there is no provision for single column floats inside a multicolumn environment, so figures and tables must be coded in-line (using, for example, the H modifier of the float package). The package is part of the tools bundle in the LaTeX required distribution.
\RequirePackage{multicol}

%% Longtable allows you to write tables that continue to the next page. You can write captions within the table (typically at the start of the table), and headers and trailers for pages of table. Longtable arranges that the columns on successive pages have the same widths. This last contrasts with the superficially similar supertabular package.
%% Longtable (unlike supertabular) modifies the output routine, and consequently won't work in a multicolumn environment (or in other circumstances where the output routine has been critically altered); it also fails in twocolumn pages.
%% This package is part of the tools bundle in the LaTeX required distribution.
\RequirePackage{longtable}

% The package allows rows and columns to be coloured, and even individual cells.
\RequirePackage{colortbl}

% The package provides an environment, tabu, which will make any sort of tabular (that doesn't need to split across pages), and an environment longtabu which provides the facilities of tabu in a modified longtable environment. (Note that this latter offers an enhancement of ltxtable.)
% The package requires the array package, and needs e-TeX to run (since array.sty is present in every conforming distribution of LaTeX, and since every publicly available LaTeX format is built using e-TeX, the requirements are provided by default on any reasonable system). The package also requires xcolor for coloured rules in tables, and colortbl for coloured cells. The longtabu environment further requires that longtable be loaded. The package itself does not load any of these packages for the user.
% The tabu environment may be used in place of tabular, tabular* and tabularx environments, as well as the array environment in maths mode. It overloads tabularx's X-column specification, allowing a width specification, alignment (l, r, c and j) and column type indication (p, m and b).
% \begin{tabu} to <dimen> specifies a target width, and \begin{tabu} spread <dimen> enlarges the environment's "natural" width.
\RequirePackage{tabu}

% The package enhances the quality of tables in LaTeX, providing extra commands as well as behind-the-scenes optimisation. Guidelines are given as to what constitutes a good table in this context. From version 1.61, the package offers longtable compatibility.
\RequirePackage{booktabs}

% The package defines new commands \Centering, \RaggedLeft, and \RaggedRight and new environments Center, FlushLeft, and FlushRight, which set ragged text and are easily configurable to allow hyphenation (the corresponding commands in LaTeX, all of whose names are lower-case, prevent hyphenation altogether).
\RequirePackage{ragged2e}

\ifDocumentmathematics
  % Typeset in-line fractions in a "nice" way. The package typesets fractions "nicely" - in the form 'a/b' (i.e., staggered with a slash between them, rather than directly one over the other). The package is distributed as part of a bundle including the units package. Nicefrac's facilities are provided, in a cleaner way, by the (experimental) xfrac package, but see also the faktor package for quotient spaces and the like.
  \RequirePackage{nicefrac}

  % fouridx - Left sub- and superscripts in maths mode
  % The package enables left subscripts and superscripts in maths mode. The sub- and superscripts are raised for optimum fitting to the symbol indexed, in such a way that left and right sub- and superscripts are set on the same level, as appropriate.
  % The package provides an alternative to the use of the \sideset command in the amsmath package.
  \RequirePackage{fouridx}

  % Place lines through maths formulae. A package to draw diagonal lines ("cancelling" a term) and arrows with limits (cancelling a term "to a value") through parts of maths formulae.
  \RequirePackage{cancel}

  % Mathtools provides a series of packages designed to enhance the appearance of documents containing a lot of mathematics. The main backbone is amsmath, so those unfamiliar with this required part of the LaTeX system will probably not find the packages very useful.
  % Mathtools provides many useful tools for mathematical typesetting. It is based on amsmath and fixes various deficiencies of amsmath and standard LaTeX. It provides:
  % * Extensible symbols, such as brackets, arrows, harpoons, etc.;
  % * Various symbols such as \coloneqq (:=);
  % * Easy creation of new tag forms;
  % * Showing equation numbers only for referenced equations;
  % * Extensible arrows, harpoons and hookarrows;
  % * Starred versions of the amsmath matrix environments for specifying the column alignment;
  % * More building blocks: multlined, cases-like environments, new gathered environments;
  % * Maths versions of \makebox, \llap, \rlap etc.;
  % * Cramped math styles; and more...
  \RequirePackage{mathtools}

  % EMPHasizing EQuations. The empheq package is part of the mathtools bundle. The package provides a visual markup extension to amsmath. The user-friendly interface allows the user to put a set of equations inside a box thus enhancing the \boxed feature of amsmath. As a side effect it's also possible to add material on both sides of the equations thus providing (and surpassing) the functionality of the cases package. Users of ntheorem will probably want to have a look at it as well, since the problem with end-of-theorem marks in gather and other environments can be circumvented using empheq.
  \RequirePackage{empheq}
\fi

% A comprehensive (SI) units package. Typesetting values with units requires care to ensure that the combined mathematical meaning of the value plus unit combination is clear. In particular, the SI units system lays down a consistent set of units with rules on how they are to be used. However, different countries and publishers have differing conventions on the exact appearance of numbers (and units). A number of LaTeX packages have been developed to provide consistent application of the various rules: SIunits, sistyle, unitsdef and units are the leading examples. The numprint package provides a large number of number-related functions, while dcolumn and rccol provide tools for typesetting tabular numbers. The siunitx package takes the best from the existing packages, and adds new features and a consistent interface. A number of new ideas have been incorporated, to fill gaps in the existing provision. The package also provides backward-compatibility with SIunits, sistyle, unitsdef and units. The aim is to have one package to handle all of the possible unit-related needs of LaTeX users. The package relies on LaTeX 3 support from the l3kernel and l3packages bundles.
\RequirePackage{siunitx}

% This package provides the command \marginnote that may be used instead of \marginpar at almost every place where \marginpar cannot be used, e.g., inside floats, footnotes, or in frames made with the framed package.
\RequirePackage{marginnote}

% A collection of ways to change the typesetting of footnotes. The package provides means of changing the layout of the footnotes themselves (including setting them in 'paragraphs' - the para option), a way to number footnotes per page (the perpage option), to make footnotes disappear in a 'moving' argument (stable option) and to deal with multiple references to footnotes from the same place (multiple option). The package also has a range of techniques for labelling footnotes with symbols rather than numbers.
% Some of the functions of the package are overlap with the functionality of other packages. The para option is also provided by the manyfoot and bigfoot packages, though those are both also portmanteau packages. (Don't be seduced by fnpara, whose implementation is improved by the present package.) The perpage option is also offered by footnpag and by the rather more general-purpose perpage
\RequirePackage{footmisc}

% The package provides four commands for vertically scaling and stretching objects. Its primary function is the ability to scale/stretch and shift one object to conform to the size of a specified second object. This feature can be useful in both equations and schematic diagrams.
% Additionally, the scaling and stretching commands offer constraints on maximum width and/or minimum aspect ratio, which are often used to preserve legibility or for the sake of general appearance. 
\RequirePackage{scalerel}

% Enumerate and itemize within paragraphs. Provides enumerate and itemize environments that can be used within paragraphs to format the items either as running text or as separate paragraphs with a preceding number or symbol. Also provides compacted versions of enumerate and itemize.
\RequirePackage{paralist}

% Set the font size relative to the current font size. The basic command of the package is \relsize, whose argument is a number of \magsteps to change size; from this are defined commands \larger, \smaller, \textlarger, etc.
\RequirePackage{relsize}

% The caption package provides many ways to customise the captions in floating environments like figure and table, and cooperates with many other packages. Facilities include rotating captions, sideways captions, continued captions (for tables or figures that come in several parts). A list of compatibility notes, for other packages, is provided in the documentation. The package also provides the "caption outside float" facility, in the same way that simpler packages like capt-of do. The package supersedes caption2.
\RequirePackage{caption}
% The package provides a means of using facilities analagous to those of the caption package, when writing captions for subfigures and the like.
% The package is distributed with caption.
\RequirePackage{subcaption}

% The appendix package provides various ways of formatting the titles of appendices. Also (sub)appendices environments are provided that can be used, for example, for per chapter/section appendices. The word 'Appendix' or similar can be prepended to the appendix number for article class documents. The word 'Appendices' or similar can be added to the table of contents before the appendices are listed. The word 'Appendices' or similar can be typeset as a \part-like heading (page) in the body. An appendices environment is provided which can be used instead of the \appendix command.
\RequirePackage{appendix}

% The package lets the user mark things to do later, in a simple and visually appealing way. The package takes several options to enable customization/finetuning of the visual appearance.
\RequirePackage{todonotes}

% Reference last page for Page N of M type footers. Reference the number of pages in your LaTeX document through the introduction of a new label which can be referenced like \pageref{LastPage} to give a reference to the last page of a document. It is particularly useful in the page footer that says: Page N of M.
\RequirePackage{lastpage}

% A package built on the standard LaTeX graphics package to perform all the different sorts of rotation one might like, including complete figures and tables with their captions.% 
% If you want continuous text (i.e., more than one page) set in landscape mode, use the lscape package instead. The rotating packages only deals in rotated boxes (or floats, which are % themselves boxes), and boxes always stay on one page.
% If you need to use the facilities of the float in the same document, load rotating.sty via rotfloat, which smooths the path between the rotating and float packages.
\RequirePackage{rotating}

% Control layout of itemize, enumerate, description. This package provides user control over the layout of the three basic list environments: enumerate, itemize and description. It supersedes both enumerate and mdwlist (providing well-structured replacements for all their funtionality), and in addition provides functions to compute the layout of labels, and to 'clone' the standard environments, to create new environments with counters of their own.
\RequirePackage{enumitem}

% Typesetting theorems (AMS style). The package facilitates the kind of theorem setup typically needed in American Mathematical Society publications. The package offers the theorem setup of the AMS document classes (amsart, amsbook, etc.) encapsulated in LaTeX package form so that it can be used with other document classes. Amsthm is part of the (required) AMS-LaTeX distribution, so should be present in any LaTeX distribution.
\RequirePackage{amsthm}

% A simple type of box for LaTeX. This small package provides a convenient input syntax for boxes that don't break their text over lines automatically, but do allow manual line breaks. The boxes shrink to the natural width of the longest line they contain
\RequirePackage{minibox}

% Generate English ordinal numbers. The command \nth{<number>} generates English ordinal numbers of the form 1st, 2nd, 3rd, 4th, etc. LaTeX package options may specify that the ordinal mark be superscripted, and that negative numbers may be treated; Plain TeX users have no access to package options, so need to redefine macros for these changes.
\RequirePackage{nth}

% Show label, ref, cite and bib keys. The showkeys package modifies the \label, \ref, \pageref, \cite and \bibitem commands so that the 'internal' key is printed, without affecting the appearance of the rest of the text, so far as is possible (the keys typically appear in the margin). The package is part of the tools bundle in the LaTeX required distribution.
\ifdraft{\RequirePackage{showkeys}}{}

% % Expand acronyms at least once. This package ensures that all acronyms used in the text are spelled out in full at least once. It also provides an environment to build a list of acronyms used. The package is compatible with pdf bookmarks. The package requires the suffix package, which in turn requires that it runs under e-TeX.
% \RequirePackage{acronym}

\ifDocumentcode
  % The package that facilitates expressive syntax highlighting in LaTeX using the powerful Pygments library. The package also provides options to customize the highlighted source code output using fancyvrb. 
  \RequirePackage{minted}
\fi

% Defines commands \counterwithin (which sets up a counter to be reset when another is incremented) and \counterwithout (which unsets such a relationship).
\RequirePackage{chngcntr}

% LaTeX's built-in two-column code finishes off a document exactly where the text stops; this will typically leave an isolated left-hand column, or a right-hand column shorter than the left-hand one. This package modifies the LaTeX output routine to make the two columns as nearly of the same length as possible.
% Only used with two-column layout (or, in general, multicols)
\RequirePackage{flushend}

% Interpretes quotation marks " and ' and correctly typesets them. Some LaTeX-Editors however provide an option for automatically replacing the correct LaTeX code for quotes, which should be preferred.
\RequirePackage{csquotes}

% The dcolumn package makes use of the array package to define a "D" column format for use in tabular environments.
% This package is part of the tools bundle in the LaTeX required distribution.
\RequirePackage{dcolumn}

% The European currency symbol for the Euro implemented in METAFONT, using the official European Commission dimensions, and providing several shapes (normal, slanted, bold, outline). The package also includes a LaTeX package which defines the macro, pre-compiled tfm files, and documentation.
\RequirePackage{eurosym}

% Use biblatex, which is the (highly configurable) successor of bibtex and biber
% biber is still used though to sort the bibliography in the background. 
\RequirePackage{biblatex}

% The command \url is a form of verbatim command that allows linebreaks at certain characters or combinations of characters, accepts reconfiguration, and can usually be used in the argument to another command. (The \urldef command provides robust commands that serve in cases when \url doesn't work in an argument.) The command is intended for email addresses, hypertext links, directories/paths, etc., which normally have no spaces, so by default the package ignores spaces in its argument. However, a package option "allows spaces", which is useful for operating systems where spaces are a common part of file names.
\RequirePackage{url}

% The hyperref package is used to handle cross-referencing commands in LaTeX to produce hypertext links in the document. The package provides backends for the \special set defined for HyperTeX DVI processors; for embedded pdfmark commands for processing by Acrobat Distiller (dvips and Y&Y's dvipsone); for Y&Y's dviwindo; for PDF control within pdfTeX and dvipdfm; for TeX4ht; and for VTeX's pdf and HTML backends.
% The package is distributed with the backref and nameref packages, which make use of the facilities of hyperref.
% The package depends on the author's kvoptions, ltxcmdsand refcount packages.
\RequirePackage{hyperref}

% Intelligent cross-referencing. The package enhances LaTeX's cross-referencing features, allowing the format of references to be determined automatically according to the type of reference. The formats used may be customised in the preamble of a document; babel support is available (though the choice of languages remains limited: currently Danish, Dutch, English, French, German, Italian, Norwegian, Russian, Spanish and Ukranian). The package also offers a means of referencing a list of references, each formatted according to its type. In such lists, it can collapse sequences of numerically-consecutive labels to a reference range.
\RequirePackage{cleveref}

% Create glossaries and lists of acronyms. The glossaries package supports acronyms and multiple glossaries, and has provision for operation in several languages (using the facilities of either babel or polyglossia). New entries are defined to have a name and description (and optionally an associated symbol). Support for multiple languages is offered, and plural forms of terms may be specified. An additional package, glossaries-accsupp, can make use of the accsupp package mechanisms for accessibility support for PDF files containing glossaries. The user may define new glossary styles, and preambles and postambles can be specified. There is provision for loading a database of terms, but only terms used in the text will be added to the relevant glossary. The package uses an indexing program to provide the actual glossary; either makeindex or xindy may serve this purpose, and a Perl script is provided to serve as interface. The package distribution also provides the mfirstuc package, for changing the first letter of a word to upper case. The package supersedes the author's glossary package (which is now obsolete), and a conversion tool is provided.
\RequirePackage{glossaries}
\RequirePackage{glossaries-extra}
\RequirePackage{glossary-longbooktabs}
\RequirePackage{glossaries-extra-stylemods}
\setabbreviationstyle[acronym]{long-short}
\renewcommand{\glossarypreamble}{\glsfindwidesttoplevelname[\currentglossary]}

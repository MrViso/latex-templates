%%%%%%%%%%%%%%%%%%%%%%%%%%%%%%%%%%%%%%%%%%%%%%%%%%%%%%%%%%%%%%%%%%%%%%%%%%%%%%%%
%% FLOAT COMMANDS                                                             %%
%%%%%%%%%%%%%%%%%%%%%%%%%%%%%%%%%%%%%%%%%%%%%%%%%%%%%%%%%%%%%%%%%%%%%%%%%%%%%%%%
% Create two new lengths for TikZ images: figurewidth and figureheight
\newlength{\figurewidth}
\newlength{\figureheight}
% Set the default lenghts of above defined lengths with aspect ratio 16:9 to width
% of 75% of line width (and height accordingly)
\setlength{\figurewidth}{0.75\linewidth}
\setlength{\figureheight}{0.421875\linewidth}

% Easily include a TikZ image with only three arguments (one required, two optional) - default aspect ratio for a tikz image is 4:3 (0.75 x 0.5625 \linewidth) [width x height]
% a) Width of rendered image
% b) Height of rendered image
% 1) Path to the file
%\newcommand{\includetikz}[3]{%
% \DeclareDocumentCommand{\includetikz}{O{0.75\linewidth} O{0.5625\linewidth} m}{%
%   \setlength{\figurewidth}{#1}%
%   \setlength{\figureheight}{#2}%
% %  \tikzsetnextfilename{#3}%
%   \input{#3.tikz}%
% }
% Deprecate `\includetikz` in favor of `\includegraphics` with the `tikzscale` package
\deprecate{includetikz}{\includegraphics}

% Easily include SVG files with only one required argument (and one optional argument)
% a) Width of the file
% 1) Path to the file
\DeclareDocumentCommand{\includesvg}{o m}{%
  \IfNoValueTF{#1}{%
    \def\svgwidth{0.75\linewidth}
  }{%
    \def\svgwidth{#1}
  }
  \executeiffilenewer{#2.svg}{#2.pdf}{
    inkscape -z -D --file=#2.svg --export-pdf=#2.pdf --export-latex
  }
  \input{#2.pdf_tex}
}

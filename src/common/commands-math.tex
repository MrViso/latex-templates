% Only load if option mathematics given
\ifDocumentmathematics
%%%%%%%%%%%%%%%%%%%%%%%%%%%%%%%%%%%%%%%%%%%%%%%%%%%%%%%%%%%%%%%%%%%%%%%%%%%%%%%%
%% MATH COMMANDS                                                              %%
%%%%%%%%%%%%%%%%%%%%%%%%%%%%%%%%%%%%%%%%%%%%%%%%%%%%%%%%%%%%%%%%%%%%%%%%%%%%%%%%

%%
% @group math
% @sample \parens{x}: Parentheses as left-right paired delimiter for argument x
% @sample \parens[\bigg]{x}: Explicitely set the size of the pairing parentheses to \latexinline|\bigg|
% @sample \parens*{x}: Automatically scale the size of the pairing parentheses
%%
\DeclarePairedDelimiterX{\parens}[1]{\lparen}{\rparen}{
  \ifblank{#1}{\:\cdot\:}{#1}%
}%
% This macro is just left for backwards compatability. Please use $\parens$ from now on
\DeclarePairedDelimiterX{\parentheses}[1]{\lparen}{\rparen}{%
  \ifblank{#1}{\:\cdot\:}{#1}%
}%

%%
% @group math
% @sample \of{x, t}: Nicer and more readable "function of" argument writing
% @sample \of[\Bigg]{x, t}: Make the pairing parentheses in size \latexinline|\bigg|
% @sample \of*{x, t}: Automatic height adjustment of the pairing parentheses
% @see https://tex.stackexchange.com/questions/2607/spacing-around-left-and-right
%%
\DeclareDocumentCommand{\of}{ s o m }{%
  \mkern-1.0mu%
  \IfBooleanTF{#1}{%
    \IfValueTF{#2}{%
      \parens*[#2]{#3}%
    }{%
      \parens*{#3}%
    }%
  }{%
    \IfValueTF{#2}{%
      \parens[#2]{#3}%
    }{%
      \parens{#3}%
    }%
  }%
}%

%%
% @group math
% @sample \overlinebm{M}: Overline the bm-typeset character M such that the overline is placed with a length matching the width of the upright version of the letter, and offsets the overstoke by 3mu to account for the italics.
%%
\DeclareDocumentCommand{\overlinebm}{ m }{%
  \ThisStyle{\ooalign{%
    $\SavedStyle\mkern3mu\overline{\phantom{\mathrm{#1}}}$\cr$\SavedStyle\bm{{#1}}$%
  }}%
}%

%%
% @group math
% @sample \overlinevect{q}: Vector notation for overlined q
% @sample \overlinevect[P]{q}: Vector notation for overlined q with pre-superscript P
% @sample \overlinevect[A][B]{q}: Vector notation for overlined q with pre-subscript A and pre-superscript B
%%
\DeclareDocumentCommand{\overlinevect}{ o o m }{%
  \IfValueTF{#1}{%
    \IfValueTF{#2}{%
      \prescript{#2}{#1}{\overlinebm{#3}}%
    }{%
      \prescript{#1}{}{\overlinebm{#3}}%
    }%
  }{%
    \overlinebm{#3}%
  }%
}%

%%
% @group math
% @sample \overlinematr{M}: Matrix notation for overlined M
% @sample \overlinematr[P]{M}: Matrix notation for overlined M with pre-superscript P
% @sample \overlinematr[A][B]{M}: Matrix notation for overlined M with pre-subscript A and pre-superscript B
% @see http://tex.stackexchange.com/questions/199789/which-bold-style-is-recommended-for-matrix-notation
%%
\DeclareDocumentCommand{\overlinematr}{ o o m }{
  \IfValueTF{#1}{%
    \IfValueTF{#2}{%
      \prescript{#2}{#1}{\overlinebm{#3}}%
    }{%
      \prescript{#1}{}{\overlinebm{#3}}%
    }%
  }{%
    \overlinebm{#3}%
  }%
}


%%
% @group math
% @sample \matr{M}: Matrix notation for M
% @sample \matr[P]{M}: Matrix notation for M with pre-superscript P
% @sample \matr[A][B]{M}: Matrix notation for M with pre-subscript A and pre-superscript B
% @see http://tex.stackexchange.com/questions/199789/which-bold-style-is-recommended-for-matrix-notation
%%
\DeclareDocumentCommand{\matr}{ o o m }{%
  \IfValueTF{#1}{%
    \IfValueTF{#2}{%
      \prescript{#2}{#1}{\bm{#3}}%
    }{%
      \prescript{#1}{}{\bm{#3}}%
    }%
  }{%
    \bm{#3}%
  }%
}%

%%
% @group math
% @sample \Dmatr{M}: Matrix notation for time-derivative of $M$
% @sample \Dmatr[P]{M}: Matrix notation for time-derivative of $M$ with pre-superscript $P$
% @sample \Dmatr[A][B]{M}: Matrix notation for time-derivative of $M$ with pre-subscript $A$ and pre-superscript $B$
% @see http://tex.stackexchange.com/questions/199789/which-bold-style-is-recommended-for-matrix-notation
%%
\DeclareDocumentCommand{\Dmatr}{ o o m }{%
  \IfValueTF{#1}{%
    \IfValueTF{#2}{%
      \prescript{#2}{#1}{\bm{\dot{#3}}}%
    }{%
      \prescript{#1}{}{\bm{\dot{#3}}}%
    }%
  }{%
    \bm{\dot{#3}}%
  }%
}%

%%
% @group math
% @sample \DDmatr{M}: Matrix notation for second time-derivative of $M$
% @sample \DDmatr[P]{M}: Matrix notation for second time-derivative of $M$ with pre-superscript $P$
% @sample \DDmatr[A][B]{M}: Matrix notation for second time-derivative of $M$ with pre-subscript $A$ and pre-superscript $B$
% @see http://tex.stackexchange.com/questions/199789/which-bold-style-is-recommended-for-matrix-notation
%%
\DeclareDocumentCommand{\DDmatr}{ o o m }{%
  \IfValueTF{#1}{%
    \IfValueTF{#2}{%
      \prescript{#2}{#1}{\bm{\ddot{#3}}}%
    }{%
      \prescript{#1}{}{\bm{\ddot{#3}}}%
    }%
  }{%
    \bm{\ddot{#3}}%
  }%
}%

%%
% @group math
% @sample \vect{q}: Vector notation for q
% @sample \vect[P]{q}: Vector notation for q with pre-superscript P
% @sample \vect[A][B]{q}: Vector notation for q with pre-subscript A and pre-superscript B
%%
\DeclareDocumentCommand{\vect}{ o o m }{%
  \IfValueTF{#1}{%
    \IfValueTF{#2}{%
      \prescript{#2}{#1}{\bm{#3}}%
    }{%
      \prescript{#1}{}{\bm{#3}}%
    }%
  }{%
    \bm{#3}%
  }%
}%

%%
% @group math
% @sample \Dvect{q}: Vector notation for time-derivative of $q$
% @sample \Dvect[P]{q}: Vector notation for time-derivative of $q$ with pre-superscript $P$
% @sample \Dvect[A][B]{q}: Vector notation for time-derivative of $q$ with pre-subscript $A$ and pre-superscript $B$
%%
\DeclareDocumentCommand{\Dvect}{ o o m }{%
  \IfValueTF{#1}{%
    \IfValueTF{#2}{%
      \prescript{#2}{#1}{\bm{\dot{#3}}}%
    }{%
      \prescript{#1}{}{\bm{\dot{#3}}}%
    }%
  }{%
    \bm{\dot{#3}}%
  }%
}%

%%
% @group math
% @sample \DDvect{q}: Vector notation for second time-derivative of $q$
% @sample \DDvect[P]{q}: Vector notation for second time-derivative of $q$ with pre-superscript $P$
% @sample \DDvect[A][B]{q}: Vector notation for second time-derivative of $q$ with pre-subscript $A$ and pre-superscript $B$
%%
\DeclareDocumentCommand{\DDvect}{ o o m }{%
  \IfValueTF{#1}{%
    \IfValueTF{#2}{%
      \prescript{#2}{#1}{\bm{\ddot{#3}}}%
    }{%
      \prescript{#1}{}{\bm{\ddot{#3}}}%
    }%
  }{%
    \bm{\ddot{#3}}%
  }%
}%

%%
% @group math
% @sample \abs{x}: Absolute value of the argument as a paired delimiter
% @see mathtools
%%
\DeclarePairedDelimiterX{\abs}[1]{\lvert}{\rvert}{%
  \ifblank{#1}{\:\cdot\:}{#1}
}%

%%
% @group math
% @sample \norm{x}: Norm of the argument as a paired delimiter
% @see mathtools
%%
\DeclarePairedDelimiterX{\norm}[1]{\lVert}{\rVert}{%
  \ifblank{#1}{\:\cdot\:}{#1}%
}%

%%
% @group math
% @sample \pow{x}: Second power of the argument
% @sample \pow[n]{x}: n-th power of the argument
% @sample \pow*{x}: Automatically adds and scales parentheses around the base $x$
% @sample \pow*[n]{x}: Automatically adds and scales parentheses around the base $x$ of power $n$
%%
\DeclareDocumentCommand{\pow}{ s O{2} m }{%
  \IfBooleanTF{#1}{%
    \parens*{#3}^{#2}%
  }{%
    #3^{#2}%
  }
}%

%%
% @group math
% @sample \skewm{m}: Commonly used skew-symmetric cross product matrix of argument m
%%
\DeclarePairedDelimiterX{\skewm}[1]{[}{]^{\mkern-2.5mu\times}}{#1}%

%%
% @group math
% @sample \crossp{a}{b}: Cross product of a and b
%%
\DeclareDocumentCommand{\crossp}{m m}{%
  {#1} \times {#2}%
}%

%%
% @group math
% @sample \diag: Math operator for a matrix as diagonal of the following vector's elements
%%
\DeclareMathOperator{\diag}{diag}%

%%
% @group math
% @sample \sign: Mathematical sign (signum) operator
%%
\DeclareMathOperator{\sign}{sign}%


%%
% @group math
% @sample \dif: Differential operator (upface d). Used in integrals for the integrand.
% @sample \dif[x]: Index the differential operator with $x$. Necessary because there is a negative space after the operator.
% @see http://tex.stackexchange.com/questions/135944/commath-and-ifinner
%%
\DeclareDocumentCommand{\dif}{ o }{%
  \operatorname{d}%
  \IfValueT{#1}{%
    _{#1}%
  }%
  \!%
}%

%%
% @group math
% @sample \dif: Derivative operator (upface D)
% @sample \dif[x]: Index the derivative operator with $x$. Necessary because there is a negative space after the operator.
%%
\DeclareDocumentCommand{\Dif}{ o }{%
  \operatorname{D}%
  \IfValueT{#1}{%
    _{#1}%
  }%
  \!%
}%

%%
% @group math
% @sample \pd{f}{x}: Partial derivative of $f$ with resepect to $x$
% @sample \pd[n]{f}{x}: n-th partial derivative of $f$ with respect to $x$
%%
\providecommand{\pd}[3][]{%
  \frac{ \partial{^{#1}}#2 }{ \partial{#3^{#1}} }%
}%

%%
% @group math
% @sample \tpd{f}{x}: Text style partial derivative of $f$ with resepect to $x$
% @sample \tpd[n]{f}{x}: Text style n-th partial derivative of $f$ with respect to $x$
%%
\providecommand{\tpd}[3][]{\mathinner{%
  \tfrac{ \partial{^{#1}}#2 }{ \partial{#3^{#1}} }%
}}%

%%
% @group math
% @sample \dpd{f}{x}: Display style partial derivative of $f$ with resepect to $x$
% @sample \dpd[n]{f}{x}: Display style n-th partial derivative of $f$ with respect to $x$
%%
\providecommand{\dpd}[3][]{\mathinner{%
  \dfrac{ \partial{^{#1}}#2 }{ \partial{#3^{#1}} }%
}}%

%%
% @group math
% @sample \md{f}{5}{x}{2}{y}{3}: Mixed $5$-th order derivative of $f$ with respect to $x$ (twofold) and $y$ (threefold).
%%
\providecommand{\md}[6]{%
  \frac{ \partial{^{#2}}#1 }{ \partial{#3^{#4}}\partial{#5^{#6}} }%
}%

%%
% @group math
% @sample \tmd{f}{5}{x}{2}{y}{3}: Text style mixed $5$-th order derivative of $f$ with respect to $x$ (twofold) and $y$ (threefold).
%%
\providecommand{\tmd}[6]{\mathinner{%
  \tfrac{ \partial{^{#2}}#1 }{ \partial{#3^{#4}}\partial{#5^{#6}} }%
}}%

%%
% @group math
% @sample \dmd{f}{5}{x}{2}{y}{3}: Display style mixed $5$-th order derivative of $f$ with respect to $x$ (twofold) and $y$ (threefold).
%%
\providecommand{\dmd}[6]{\mathinner{%
  \dfrac{ \partial{^{#2}}#1 }{ \partial{#3^{#4}}\partial{#5^{#6}} }%
}}%

%%
% @group math
% @sample \od{f}{x}: Ordinary derivative of $f$ with respect to $x$
% @sample \od[n]{f}{x}: $n$-th order ordinary derivative of $f$ with respect to $x$
%%
\providecommand{\od}[3][]{%
  \frac{ \dif{^{#1}}#2 }{ \dif{#3^{#1}} }%
}%

%%
% @group math
% @sample \od{f}{x}: Text style ordinary derivative of $f$ with respect to $x$
% @sample \od[n]{f}{x}: Text style $n$-th order ordinary derivative of $f$ with respect to $x$
%%
\providecommand{\tod}[3][]{\mathinner{%
  \tfrac{ \dif{^{#1}}#2 }{ \dif{#3^{#1}} }%
}}%

%%
% @group math
% @sample \od{f}{x}: Display style ordinary derivative of $f$ with respect to $x$
% @sample \od[n]{f}{x}: Display style $n$-th order ordinary derivative of $f$ with respect to $x$
%%
\providecommand{\dod}[3][]{\mathinner{%
  \dfrac{ \dif{^{#1}}#2 }{ \dif{#3^{#1}} }%
}}%


%%
% @group math
% @sample \dif{x}: Total differential of $x$
%%
\DeclareDocumentCommand{\td}{ m }{%
  \dif{#1}%
}%

%%
% @group math
% @sample \Re: Operator for real value of complex number
%%
\renewcommand{\Re}{\operatorname{Re}}%

%%
% @group math
% @sample \Im: Operator for imaginary value of complex number
%%
\renewcommand{\Im}{\operatorname{Im}}%

%%
% @group math
% @sample \transp{A}: Transpose of argument $A$ e.g., matrix or vector transpose
% @sample \transp[\mkern-2.5mu]{x}: Set the spacing of the superscript index for the transpose symbol manually. This may be useful in case the superscript is too close or too far away from its base
%%
\DeclareDocumentCommand{\transp}{ s O{\mkern-1.5mu} m }{%
  \IfBooleanTF{#1}{%
    \parentheses{#3}^{#2\mbox{\textscale{0.6}{$\top$}}}%%
  }{%
    #3^{#2\mbox{\textscale{0.6}{$\top$}}}%%
  }%
}%

%%
% @group math
% @sample \trafom{A}{B}: Transformation matrix $\matr{T}$ from coordinate system $\frame{A}$ to $\frame{B}$
% @sample \trafom[Q]{A}{B}: Denote transformation matrix with $\matr{Q}$ instead of $\matr{T}$
%%
\DeclareDocumentCommand{\trafom}{ O{T} m m }{%
  \matr[#2][#3]{#1}%
}%

%%
% @group math
% @sample \conj{m}: Conjugate complex of the argument
%%
\DeclareDocumentCommand{\conj}{ s m }{%
  \overline{#1}%
}%

%%
% @group math
% @sample \hermconj{x}: Hermitian conjugate of matrix
% @sample \hermconj[\mkern-2.5mu]{x}: Set the spacing of the superscript index for the Hermitian transpose manually. This may be useful in case the superscript is too close or too far away from its base
%%
\DeclareDocumentCommand{\hermconj}{ s O{\mkern-1.5mu} m }{%
  \IfBooleanTF{#1}{%
    \parentheses{#3}^{#2\mbox{\textscale{0.6}{$\mathsf{H}$}}}%
  }{%
    #3^{#2\mbox{\textscale{0.6}{$\mathsf{H}$}}}%
  }%
}%


%%
% @group math
% @sample \inv{A}: Element inverse of A - usually used for matrix inverse
% @sample \inv*{A}: Automatically wraps the argument in parentheses
% @sample \inv*[\bigg]{A}: Sets the scaling of the parentheses around the argument to bigg
%%
\DeclareDocumentCommand{\inv}{ s o m }{%
  \IfBooleanTF{#1}{%
    \IfValueTF{#2}{
      \parentheses[#2]{#3}^{-1}%
    }{%
      \parentheses{#3}^{-1}%
    }%
  }{%
    #3^{-1}%
  }%
}%

%%
% @group math
% @sample \dotp{a}{b}: Dot product of two arguments $a$ and $b$
%%
\DeclareDocumentCommand{\dotp}{ m m }{%
  {#1} \cdot {#2}%
}%

%%
% @group math
% @sample \evec{x}: Unit axis coordinate vector along the $x$-axis
% @sample \evec[\xi]{x}: Name the unit axis coordinate vector $\xi$ rather than $e$
%%
\DeclareDocumentCommand{\evec}{ O{e} m }{%
  \hat{ \vect{#1} }_{\mc{#2}}%
}%

%%
% @group math
% @sample \arctantwo: Math operator for the arcus tangent 2
%%
\DeclareMathOperator{\arctantwo}{atan2}%

%%
% @group math
% @sample \arctant{y}{x}: Arcus tangent 2 of the two arguments
% @sample \arctant[\bigg]{y}{x}: Arcus tangent 2 with pairing parentheses scaled to size \latexinline|\bigg|
% @sample \arctant*{\frac{a}{b}}{x}: Arcus tangent 2 with automatically scaled pairing parentheses
%%
\DeclareDocumentCommand{\arctant}{ s o m m }{%
  \arctantwo\, %
  \IfBooleanTF{#1}{%
    \of*{#3, #4}%
  }{%
    \IfValueTF{#2}{%
      \of[#2]{#3, #4}%
    }{%
      \of{#3, #4}%
    }
  }%
}%

%%
% @group math
% @sample \trace: Math operator for the trace of a matrix
%%
\DeclareMathOperator{\trace}{tr}%

%%
% @group math
% @sample \irange{m}: Range of numbers from $1$ to $m$
% @sample \irange{m}{j}: Range of numbers with running variable $j$
% @sample \irange[2]{m}: Increment running variable by $2$ with each step
% @sample \irange[2]{m}{j}: Running variable now is $j$ and increases by $2$ each step
%%
\NewDocumentCommand{\irange}{o m g}{%
  \def\runner{\IfValueTF{#3}{#3}{i}}%
  \IfValueTF{#1}{%
    {\runner = 1, #1, \ldots, #2}%
  }{%
    {\runner = 1, \ldots, #2}%
  }%
}%

%%
% @group math
% @sample \ms{avg}: Well-scaled math subscripts of $\mc{avg}$
%%
\DeclareDocumentCommand{\ms}{ m }{%
  {\mbox{\textscale{0.6}{\mc{#1}}}}%
}%

%%
% @group math
% @sample \mc{avg}: Correct type setting of math constant for textual indices like "std" or "coeff" -- Do not use for variable indices like "i" or "j"
%%
\DeclareDocumentCommand{\mc}{ m }{%
  \text{#1}%
}%

%%
% @group math
% @sample \rotscal: Scalar rotation matrix which may be used as a rotation matrix element $\rotscal_{ij}$
%%
\DeclareDocumentCommand{\rotscal}{ }{%
  R%
}%

%%
% @group math
% @sample \rotmatr: Rotation matrix
% @sample \rotmatr[P]: Rotation matrix from coordinate system P into world system O
% @sample \rotmatr[A][B]: Rotation matrix from coordinate system A into system B
%%
\DeclareDocumentCommand{\rotmatr}{ o o }{%
  \IfValueTF{#1}{%
    \IfValueTF{#2}{%
      \matr[#1][#2]{\rotscal}%
    }{%
      \matr[][#1]{\rotscal}%
    }%
  }{%
    \matr{\rotscal}%
  }%
}%

%%
% @group math
% @sample \rotx: Rotation matrix about $x$-axis
% @sample \rotx{\alpha}: Rotation matrix about $x$-axis for angle $\alpha$
% @sample \rotx[\bigg]{\alpha}: Adjust size of the pairing parentheses of $\alpha$ to be larger
% @sample \rotx*{\alpha}: Automatically scale the pairing parentheses of $\alpha$
%%
\DeclareDocumentCommand{\rotx}{ s o g }{%
  \rotmatr_{\ms{x}}%
  \IfValueT{#3}{%
    \IfBooleanTF{#1}{%
      \of*{#3}%
    }{%
      \IfValueTF{#2}{%
        \of[#2]{#3}%
      }{%
        \of{#3}%
      }%
    }%
  }%
}%

%%
% @group math
% @sample \roty: Rotation matrix about $y$-axis
% @sample \roty{\beta}: Rotation matrix about $y$-axis for angle $\beta$
% @sample \roty[\bigg]{\beta}: Adjust size of the pairing parentheses of $\beta$ to be larger
% @sample \roty*{\beta}: Automatically scale the pairing parentheses of $\beta$
%%
\DeclareDocumentCommand{\roty}{ s o g }{%
  \rotmatr_{\ms{y}}%
  \IfValueT{#3}{%
    \IfBooleanTF{#1}{%
      \of*{#3}%
    }{%
      \IfValueTF{#2}{%
        \of[#2]{#3}%
      }{%
        \of{#3}%
      }%
    }%
  }%
}%

%%
% @group math
% @sample \rotz: Rotation matrix about $z$-axis
% @sample \rotz{\gamma}: Rotation matrix about $z$-axis for angle $\gamma$
% @sample \rotz[\bigg]{\gamma}: Adjust size of the pairing parentheses of $\gamma$ to be larger
% @sample \rotz*{\gamma}: Automatically scale the pairing parentheses of $\gamma$
%%
\DeclareDocumentCommand{\rotz}{ s o g }{%
  \rotmatr_{\ms{z}}%
  \IfValueT{#3}{%
    \IfBooleanTF{#1}{%
      \of*{#3}%
    }{%
      \IfValueTF{#2}{%
        \of[#2]{#3}%
      }{%
        \of{#3}%
      }%
    }%
  }%
}%

%%
% @group math
% @sample \imagu: Proper imaginary unit $\imagu$
%%
\DeclareMathOperator{\imagu}{\mathrm{i}}%

%%
% @group math
% @sample \imgu: Imaginary unit as command
%%
\DeclareDocumentCommand{\imgu}{ }{%
  {\imagu\mkern1mu}%
}%

%%
% @group math
% @sample \arcsec: Math operator for inverse secant
%%
\DeclareMathOperator{\arcsec}{arcsec}%

%%
% @group math
% @sample \arccot: Math operator for inverse cotangent
%%
\DeclareMathOperator{\arccot}{arccot}%

%%
% @group math
% @sample \arccsc: Math operator for inverse cosecant
%%
\DeclareMathOperator{\arccsc}{arccsc}%

%%
% @group math
% @sample \sech: Math operator for hyperbolic secant
%%
\DeclareMathOperator{\sech}{sech}%

%%
% @group math
% @sample \csch: Math operator for hyperbolic cosecant
%%
\DeclareMathOperator{\csch}{csch}%

%%
% @group math
% @sample \arccosh: Math operator for inverse hyperbolic cosine
%%
\DeclareMathOperator{\arccosh}{arccosh}%

%%
% @group math
% @sample \arcsinh: Math operator for inverse hyperbolic sinus
%%
\DeclareMathOperator{\arcsinh}{arcsinh}%

%%
% @group math
% @sample \arcsech: Math operator for inverse hyperbolic secant
%%
\DeclareMathOperator{\arcsech}{arcsech}%

%%
% @group math
% @sample \arccsch: Math operator for inverse hyperbolic cosecent
%%
\DeclareMathOperator{\arccsch}{arccsch}%

%%
% @group math
% @sample \arctanh: Math operator for inverse hyperbolic tangent
%%
\DeclareMathOperator{\arctanh}{arctanh}%

%%
% @group math
% @sample \arccoth: Math operator for inverse hyperbolic cotangent
%%
\DeclareMathOperator{\arccoth}{arccoth}%


%%
% @group math
% @sample \prtrig{\sin}{\beta}: Typeset trigonometric function $\sin$ with its functional argument $\beta$ in parentheses
% @sample \prtrig[\bigg]{\sin}{\beta}: Typeset trigonometric function with pairing parentheses scaled to size \latexinline|\bigg|
% @sample \prtrig*{\sin}{\beta}: Automatic adjustment of the parentheses sizes
%%
\DeclareDocumentCommand{\prtrig}{ s o m m }{%
  #3{%
    \IfBooleanTF{#1}{%
      \of*{#4}%
    }{%
      \IfValueTF{#2}{%
        \of[#2]{#4}%
      }{%
        \of{#4}%
      }%
    }%
  }%
}%

%%
% @group math
% @sample \psin{x}: Sine of $x$ with parentheses around $x$
% @sample \psin[\bigg]{x}: Sine of $x$ with pairing parentheses' heights to \latexinline|\bigg|
% @sample \psin*{x}: Sine of $x$ with automatically height-adjusted parentheses around $x$
%%
\DeclareDocumentCommand{\psin}{ s o m }{%
  \IfBooleanTF{#1}{%
    \prtrig*{\sin}{#3}%
  }{%
    \IfValueTF{#2}{%
      \prtrig[#2]{\sin}{#3}%
    }{%
      \prtrig{\sin}{#3}%
    }
  }%
}%

%%
% @group math
% @sample \pcos{x}: Cosine of $x$ with parentheses around $x$
% @sample \pcos[\bigg]{x}: Cosine of $x$ with pairing parentheses' heights to \latexinline|\bigg|
% @sample \pcos*{x}: Cosine of $x$ with automatically height-adjusted parentheses around $x$
%%
\DeclareDocumentCommand{\pcos}{ s o m }{%
  \IfBooleanTF{#1}{%
    \prtrig*{\cos}{#3}%
  }{%
    \IfValueTF{#2}{%
      \prtrig[#2]{\cos}{#3}%
    }{%
      \prtrig{\cos}{#3}%
    }
  }%
}%

%%
% @group math
% @sample \ptan{x}: Tanget of $x$ with parentheses around $x$
% @sample \ptan[\bigg]{x}: Tanget of $x$ with pairing parentheses' heights to \latexinline|\bigg|
% @sample \ptan*{x}: Tanget of $x$ with automatically height-adjusted parentheses around $x$
%%
\DeclareDocumentCommand{\ptan}{ s o m }{%
  \IfBooleanTF{#1}{%
    \prtrig*{\tan}{#3}%
  }{%
    \IfValueTF{#2}{%
      \prtrig[#2]{\tan}{#3}%
    }{%
      \prtrig{\tan}{#3}%
    }
  }%
}%

%%
% @group math
% @sample \psec{x}: Secant of $x$ with parentheses around $x$
% @sample \psec[\bigg]{x}: Secant of $x$ with pairing parentheses' heights to \latexinline|\bigg|
% @sample \psec*{x}: Secant of $x$ with automatically height-adjusted parentheses around $x$
%%
\DeclareDocumentCommand{\psec}{ s o m }{%
  \IfBooleanTF{#1}{%
    \prtrig*{\sec}{#3}%
  }{%
    \IfValueTF{#2}{%
      \prtrig[#2]{\sec}{#3}%
    }{%
      \prtrig{\sec}{#3}%
    }
  }%
}%

%%
% @group math
% @sample \pcsc{x}: Cosecant of $x$ with parentheses around $x$
% @sample \pcsc[\bigg]{x}: Cosecant of $x$ with pairing parentheses' heights to \latexinline|\bigg|
% @sample \pcsc*{x}: Cosecant of $x$ with automatically height-adjusted parentheses around $x$
%%
\DeclareDocumentCommand{\pcsc}{ s o m }{%
  \IfBooleanTF{#1}{%
    \prtrig*{\csc}{#3}%
  }{%
    \IfValueTF{#2}{%
      \prtrig[#2]{\csc}{#3}%
    }{%
      \prtrig{\csc}{#3}%
    }
  }%
}%

%%
% @group math
% @sample \pcot{x}: Cotangent of $x$ with parentheses around $x$
% @sample \pcot[\bigg]{x}: Cotangent of $x$ with pairing parentheses' heights to \latexinline|\bigg|
% @sample \pcot*{x}: Cotangent of $x$ with automatically height-adjusted parentheses around $x$
%%
\DeclareDocumentCommand{\pcot}{ s o m }{%
  \IfBooleanTF{#1}{%
    \prtrig*{\cot}{#3}%
  }{%
    \IfValueTF{#2}{%
      \prtrig[#2]{\cot}{#3}%
    }{%
      \prtrig{\cot}{#3}%
    }
  }%
}%


%%
% @group math
% @sample \parcsin{x}: Inverse sine of $x$ with parentheses around $x$
% @sample \parcsin[\bigg]{x}: Inverse sine of $x$ with pairing parentheses' heights to \latexinline|\bigg|
% @sample \parcsin*{x}: Inverse sine of $x$ with automatically height-adjusted parentheses around $x$
%%
\DeclareDocumentCommand{\parcsin}{ s o m }{%
  \IfBooleanTF{#1}{%
    \prtrig*{\arcsin}{#3}%
  }{%
    \IfValueTF{#2}{%
      \prtrig[#2]{\arcsin}{#3}%
    }{%
      \prtrig{\arcsin}{#3}%
    }
  }%
}%

%%
% @group math
% @sample \parccos{x}: Inverse cosine of $x$ with parentheses around $x$
% @sample \parccos[\bigg]{x}: Inverse cosine of $x$ with pairing parentheses' heights to \latexinline|\bigg|
% @sample \parccos*{x}: Inverse cosine of $x$ with automatically height-adjusted parentheses around $x$
%%
\DeclareDocumentCommand{\parccos}{ s o m }{%
  \IfBooleanTF{#1}{%
    \prtrig*{\arccos}{#3}%
  }{%
    \IfValueTF{#2}{%
      \prtrig[#2]{\arccos}{#3}%
    }{%
      \prtrig{\arccos}{#3}%
    }
  }%
}%

%%
% @group math
% @sample \parctan{x}: Inverse tangent of $x$ with parentheses around $x$
% @sample \parctan[\bigg]{x}: Inverse tangent of $x$ with pairing parentheses' heights to \latexinline|\bigg|
% @sample \parctan*{x}: Inverse tangent of $x$ with automatically height-adjusted parentheses around $x$
%%
\DeclareDocumentCommand{\parctan}{ s o m }{%
  \IfBooleanTF{#1}{%
    \prtrig*{\arctan}{#3}%
  }{%
    \IfValueTF{#2}{%
      \prtrig[#2]{\arctan}{#3}%
    }{%
      \prtrig{\arctan}{#3}%
    }
  }%
}%

%%
% @group math
% @sample \parcsec{x}: Inverse secant of $x$ with parentheses around $x$
% @sample \parcsec[\bigg]{x}: Inverse secant of $x$ with pairing parentheses' heights to \latexinline|\bigg|
% @sample \parcsec*{x}: Inverse secant of $x$ with automatically height-adjusted parentheses around $x$
%%
\DeclareDocumentCommand{\parcsec}{ s o m }{%
  \IfBooleanTF{#1}{%
    \prtrig*{\arcsec}{#3}%
  }{%
    \IfValueTF{#2}{%
      \prtrig[#2]{\arcsec}{#3}%
    }{%
      \prtrig{\arcsec}{#3}%
    }
  }%
}%

%%
% @group math
% @sample \parccsc{x}: Inverse cosecant of $x$ with parentheses around $x$
% @sample \parccsc[\bigg]{x}: Inverse cosecant of $x$ with pairing parentheses' heights to \latexinline|\bigg|
% @sample \parccsc*{x}: Inverse cosecant of $x$ with automatically height-adjusted parentheses around $x$
%%
\DeclareDocumentCommand{\parccsc}{ s o m }{%
  \IfBooleanTF{#1}{%
    \prtrig*{\arccsc}{#3}%
  }{%
    \IfValueTF{#2}{%
      \prtrig[#2]{\arccsc}{#3}%
    }{%
      \prtrig{\arccsc}{#3}%
    }
  }%
}%

%%
% @group math
% @sample \parccot{x}: Inverse cotangent of $x$ with parentheses around $x$
% @sample \parccot[\bigg]{x}: Inverse cotangent of $x$ with pairing parentheses' heights to \latexinline|\bigg|
% @sample \parccot*{x}: Inverse cotangent of $x$ with automatically height-adjusted parentheses around $x$
%%
\DeclareDocumentCommand{\parccot}{ s o m }{%
  \IfBooleanTF{#1}{%
    \prtrig*{\arccot}{#3}%
  }{%
    \IfValueTF{#2}{%
      \prtrig[#2]{\arccot}{#3}%
    }{%
      \prtrig{\arccot}{#3}%
    }
  }%
}%

%%
% @group math
% @sample \psinh{x}: Hyperbolic sine of $x$ with parentheses around $x$
% @sample \psinh[\bigg]{x}: Hyperbolic sine of $x$ with pairing parentheses' heights to \latexinline|\bigg|
% @sample \psinh*{x}: Hyperbolic sine of $x$ with automatically height-adjusted parentheses around $x$
%%
\DeclareDocumentCommand{\psinh}{ s o m }{%
  \IfBooleanTF{#1}{%
    \prtrig*{\sinh}{#3}%
  }{%
    \IfValueTF{#2}{%
      \prtrig[#2]{\sinh}{#3}%
    }{%
      \prtrig{\sinh}{#3}%
    }
  }%
}%

%%
% @group math
% @sample \pcosh{x}: Hyperbolic cosine of $x$ with parentheses around $x$
% @sample \pcosh[\bigg]{x}: Hyperbolic cosine of $x$ with pairing parentheses' heights to \latexinline|\bigg|
% @sample \pcosh*{x}: Hyperbolic cosine of $x$ with automatically height-adjusted parentheses around $x$
%%
\DeclareDocumentCommand{\pcosh}{ s o m }{%
  \IfBooleanTF{#1}{%
    \prtrig*{\cosh}{#3}%
  }{%
    \IfValueTF{#2}{%
      \prtrig[#2]{\cosh}{#3}%
    }{%
      \prtrig{\cosh}{#3}%
    }
  }%
}%

%%
% @group math
% @sample \ptanh{x}: Hyperbolic tangent of $x$ with parentheses around $x$
% @sample \ptanh[\bigg]{x}: Hyperbolic tanget of $x$ with pairing parentheses' heights to \latexinline|\bigg|
% @sample \ptanh*{x}: Hyperbolic tangent of $x$ with automatically height-adjusted parentheses around $x$
%%
\DeclareDocumentCommand{\ptanh}{ s o m }{%
  \IfBooleanTF{#1}{%
    \prtrig*{\tanh}{#3}%
  }{%
    \IfValueTF{#2}{%
      \prtrig[#2]{\tanh}{#3}%
    }{%
      \prtrig{\tanh}{#3}%
    }
  }%
}%

%%
% @group math
% @sample \pcoth{x}: Hyperbolic cotangent of $x$ with parentheses around $x$
% @sample \pcoth[\bigg]{x}: Hyperbolic cotangent of $x$ with pairing parentheses' heights to \latexinline|\bigg|
% @sample \pcoth*{x}: Hyperbolic cotangent of $x$ with automatically height-adjusted parentheses around $x$
%%
\DeclareDocumentCommand{\pcoth}{ s o m }{%
  \IfBooleanTF{#1}{%
    \prtrig*{\coth}{#3}%
  }{%
    \IfValueTF{#2}{%
      \prtrig[#2]{\coth}{#3}%
    }{%
      \prtrig{\coth}{#3}%
    }
  }%
}%

%%
% @group math
% @sample \pexp{x}: Exponential function of $x$ with parentheses around $x$
% @sample \pexp[\bigg]{x}: Expontential function of $x$ with pairing parentheses' heights to \latexinline|\bigg|
% @sample \pexp*{x}: Exponential function of $x$ with automatically height-adjusted parentheses around $x$
%%
\DeclareDocumentCommand{\pexp}{ s o m }{%
  \IfBooleanTF{#1}{%
    \prtrig*{\exp}{#3}%
  }{%
    \IfValueTF{#2}{%
      \prtrig[#2]{\exp}{#3}%
    }{%
      \prtrig{\exp}{#3}%
    }
  }%
}%

%%
% @group math
% @sample \plog{x}: Logarithm base $10$ of $x$ with parentheses around $x$
% @sample \plog[\bigg]{x}: Logarithm base $10$ of $x$ with pairing parentheses' heights to \latexinline|\bigg|
% @sample \plog*{x}: Logarithm base $10$ of $x$ with automatically height-adjusted parentheses around $x$
%%
\DeclareDocumentCommand{\plog}{ s o m }{%
  \IfBooleanTF{#1}{%
    \prtrig*{\log}{#3}%
  }{%
    \IfValueTF{#2}{%
      \prtrig[#2]{\log}{#3}%
    }{%
      \prtrig{\log}{#3}%
    }
  }%
}%

%%
% @group math
% @sample \pln{x}: Logarithm base $2$ of $x$ with parentheses around $x$
% @sample \pln[\bigg]{x}: Logarithm base $2$ of $x$ with pairing parentheses' heights to \latexinline|\bigg|
% @sample \pln*{x}: Logarithm base $2$ of $x$ with automatically height-adjusted parentheses around $x$
%%
\DeclareDocumentCommand{\pln}{ s o m }{%
  \IfBooleanTF{#1}{%
    \prtrig*{\ln}{#3}%
  }{%
    \IfValueTF{#2}{%
      \prtrig[#2]{\ln}{#3}%
    }{%
      \prtrig{\ln}{#3}%
    }
  }%
}%

%%
% @group math
% @sample \plg{x}: Logarithm base $e$ of $x$ with parentheses around $x$
% @sample \plg[\bigg]{x}: Logarithm base $e$ of $x$ with pairing parentheses' heights to \latexinline|\bigg|
% @sample \plg*{x}: Logarithm base $e$ of $x$ with automatically height-adjusted parentheses around $x$
%%
\DeclareDocumentCommand{\plg}{ s o m }{%
  \IfBooleanTF{#1}{%
    \prtrig*{\lg}{#3}%
  }{%
    \IfValueTF{#2}{%
      \prtrig[#2]{\lg}{#3}%
    }{%
      \prtrig{\lg}{#3}%
    }
  }%
}%

\fi
% END Only load if option mathematics given

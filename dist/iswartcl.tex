\documentclass[%
  draft,%
  oneside,%
  ngerman,%
  english,% last language set will be the main document language
]{iswartcl}





%%%%%%%%%%%%%%%%%%%%%%%%%%%%%%%%%%%%%%%%%%%%%%%%%%%%%%%%%%%%%%%%%%%%%%%%%%%%%%%%
% START setting document properties
% Set author(s) names
\author{Philipp Tempel \and Second Author}
% Short author names
\shortauthor{Ph.\@ Tempel \and S.\@ Author}

% Set the date this document is printed/published
% Avoid typing in manually, either use `\today` or `\printdate{yyyy-mm-dd}`
\date{\printdate{2017-10-02}}

% Title of document
\title{This is the Long Long Title of My Document Which is Going to Deal with Something Cool}
% Define a short title if the main title is too long and truncated in parts of the document
\shorttitle{This is the Shorter Title of My Document}
% END setting document properties
%%%%%%%%%%%%%%%%%%%%%%%%%%%%%%%%%%%%%%%%%%%%%%%%%%%%%%%%%%%%%%%%%%%%%%%%%%%%%%%%



%%%%%%%%%%%%%%%%%%%%%%%%%%%%%%%%%%%%%%%%%%%%%%%%%%%%%%%%%%%%%%%%%%%%%%%%%%%%%%%%
% START used defined content

% Put all your custom code (packages, styles, macros, etc.)

% This package is only included for demo purposes
\usepackage[math]{blindtext}

% END user defined content
%%%%%%%%%%%%%%%%%%%%%%%%%%%%%%%%%%%%%%%%%%%%%%%%%%%%%%%%%%%%%%%%%%%%%%%%%%%%%%%%





%%%%%%%%%%%%%%%%%%%%%%%%%%%%%%%%%%%%%%%%%%%%%%%%%%%%%%%%%%%%%%%%%%%%%%%%%%%%%%%%
% BEGIN document
\begin{document}



%%%%%%%%%%%%%%%%%%%%%%%%%%%%%%%%%%%%%%%%%%%%%%%%%%%%%%%%%%%%%%%%%%%%%%%%%%%%%%%%
% START frontmatter
\frontmatter



%%%%%%%%%%%%%%%%%%%%%%%%%%%%%%%%%%%%%%%%%%%%%%%%%%%%%%%%%%%%%%%%%%%%%%%%%%%%%%%%
% START title page in main language
\maketitle
% END title page in main language
%%%%%%%%%%%%%%%%%%%%%%%%%%%%%%%%%%%%%%%%%%%%%%%%%%%%%%%%%%%%%%%%%%%%%%%%%%%%%%%%



%%%%%%%%%%%%%%%%%%%%%%%%%%%%%%%%%%%%%%%%%%%%%%%%%%%%%%%%%%%%%%%%%%%%%%%%%%%%%%%%
% START abstract in main language
\begin{abstract}
  \Blindtext[2][1]
\end{abstract}
% END abstract in main language
%%%%%%%%%%%%%%%%%%%%%%%%%%%%%%%%%%%%%%%%%%%%%%%%%%%%%%%%%%%%%%%%%%%%%%%%%%%%%%%%



%%%%%%%%%%%%%%%%%%%%%%%%%%%%%%%%%%%%%%%%%%%%%%%%%%%%%%%%%%%%%%%%%%%%%%%%%%%%%%%%
% START lists of flaots

% Table of all contents
\tableofcontents

% List of all figures
\listoffigures

% List of all tables
\listoftables

% List of todos (only printed in non-final mode)
% \listoftodos

% END list of floats
%%%%%%%%%%%%%%%%%%%%%%%%%%%%%%%%%%%%%%%%%%%%%%%%%%%%%%%%%%%%%%%%%%%%%%%%%%%%%%%%



%%%%%%%%%%%%%%%%%%%%%%%%%%%%%%%%%%%%%%%%%%%%%%%%%%%%%%%%%%%%%%%%%%%%%%%%%%%%%%%%
% START mainmatter
\mainmatter



%%%%%%%%%%%%%%%%%%%%%%%%%%%%%%%%%%%%%%%%%%%%%%%%%%%%%%%%%%%%%%%%%%%%%%%%%%%%%%%%
% START actual document content


%%%%%%%%%%%%%%%%%%%%%%%%%%%%%%%%%%%%%%%%
% START blind document
% THIS IS JUST SOME PLACEHOLDER BLINDDOCUMENT CODE. REMOVE BEFORE CREATING YOUR%
% DOCUMENT
\Blinddocument

\begin{figure}
  \centering
  \smaller[1]
  \includegraphics[width=0.75\linewidth]{example-image-a}
  \caption{%
    A single figure
  }
  \label{fig:single-figure}
\end{figure}

\begin{figure}
  \centering
  \smaller[1]
  \begin{subfigure}[t]{0.49\linewidth}
    \centering
    \includegraphics[width=\linewidth]{example-image-a}
    \caption{%
      Subfigured example image a.
    }
    \label{fig:subfigured-1x2:a}
  \end{subfigure}%
  \hfill%
  \begin{subfigure}[t]{0.49\linewidth}
    \centering
    \includegraphics[width=\linewidth]{example-image-b}
    \caption{%
      Subfigured example image b.
    }
    \label{fig:subfigured-1x2:b}
  \end{subfigure}%
  \caption{%
    A $1 \times 2$ subfigured figure.
  }
  \label{fig:subfigured-1x2}
\end{figure}

\begin{figure}
  \centering
  \smaller[1]
  \begin{subfigure}[t]{0.49\linewidth}
    \centering
    \includegraphics[width=\linewidth]{example-image-a}
    \caption{%
      Subfigured example image a.
    }
    \label{fig:subfigured-2x2:a}
  \end{subfigure}%
  \hfill%
  \begin{subfigure}[t]{0.49\linewidth}
    \centering
    \includegraphics[width=\linewidth]{example-image-b}
    \caption{%
      Subfigured example image b.
    }
    \label{fig:subfigured-2x2:b}
  \end{subfigure}%
  \hfill%
  \begin{subfigure}[t]{0.49\linewidth}
    \centering
    \includegraphics[width=\linewidth]{example-image-c}
    \caption{%
      Subfigured example image c.
    }
    \label{fig:subfigured-2x2:c}
  \end{subfigure}%
  \hfill%
  \begin{subfigure}[t]{0.49\linewidth}
    \centering
    \includegraphics[width=\linewidth]{example-image}
    \caption{%
      Subfigured example image d.
    }
    \label{fig:subfigured-2x2:d}
  \end{subfigure}%
  \caption{%
    A $2 \times 2$ subfigured figure comprised of \subref{fig:subfigured-2x2:a}, \subref{fig:subfigured-2x2:b}, \subref{fig:subfigured-2x2:c}, and \subref{fig:subfigured-2x2:d}.
  }
  \label{fig:subfigured-2x2}
\end{figure}

We have \cref{fig:single-figure} and \cref{fig:subfigured-1x2} as well as \cref{fig:subfigured-1x2:a,fig:subfigured-1x2:b}.
Oh, did you see the $4 \times 4$ figure matrix in \cref{fig:subfigured-2x2} composed of \cref{fig:subfigured-2x2:a,fig:subfigured-2x2:b,fig:subfigured-2x2:c,fig:subfigured-2x2:d}?

\Blinddocument
% END blind document
%%%%%%%%%%%%%%%%%%%%%%%%%%%%%%%%%%%%%%%%


% END actual document content
%%%%%%%%%%%%%%%%%%%%%%%%%%%%%%%%%%%%%%%%%%%%%%%%%%%%%%%%%%%%%%%%%%%%%%%%%%%%%%%%



\end{document}
% END document
%%%%%%%%%%%%%%%%%%%%%%%%%%%%%%%%%%%%%%%%%%%%%%%%%%%%%%%%%%%%%%%%%%%%%%%%%%%%%%%%

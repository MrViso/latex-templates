% \iffalse meta-comment
%
% Copyright (C) 2019 by Philipp Tempel <latex@philipptempel.me>
% -------------------------------------------------------
% 
% This file may be distributed and/or modified under the
% conditions of the LaTeX Project Public License, either version 1.3
% of this license or (at your option) any later version.
% The latest version of this license is in:
%
%    http://www.latex-project.org/lppl.txt
%
% and version 1.3 or later is part of all distributions of LaTeX 
% version 2005/12/01 or later.
%
% \fi
%
% \iffalse
%<*driver>
\ProvidesFile{ustuttappendix.dtx}
%</driver>
%<class>\NeedsTeXFormat{LaTeX2e}[1999/12/01]
%<class>\ProvidesClass{ustuttappendix}
%<*class>
    [2019/12/07 v1.0.0 University of Stuttgart Appendix package]
%</class>
%
%<*driver>
\documentclass{ltxdoc}
\EnableCrossrefs         
\CodelineIndex
\RecordChanges
\begin{document}
  \DocInput{ustuttappendix.dtx}
\end{document}
%</driver>
% \fi
%
% \CheckSum{0}
%
% \CharacterTable
%  {Upper-case    \A\B\C\D\E\F\G\H\I\J\K\L\M\N\O\P\Q\R\S\T\U\V\W\X\Y\Z
%   Lower-case    \a\b\c\d\e\f\g\h\i\j\k\l\m\n\o\p\q\r\s\t\u\v\w\x\y\z
%   Digits        \0\1\2\3\4\5\6\7\8\9
%   Exclamation   \!     Double quote  \"     Hash (number) \#
%   Dollar        \$     Percent       \%     Ampersand     \&
%   Acute accent  \'     Left paren    \(     Right paren   \)
%   Asterisk      \*     Plus          \+     Comma         \,
%   Minus         \-     Point         \.     Solidus       \/
%   Colon         \:     Semicolon     \;     Less than     \<
%   Equals        \=     Greater than  \>     Question mark \?
%   Commercial at \@     Left bracket  \[     Backslash     \\
%   Right bracket \]     Circumflex    \^     Underscore    \_
%   Grave accent  \`     Left brace    \{     Vertical bar  \|
%   Right brace   \}     Tilde         \~}
%
%
% \changes{v1.0}{2019/12/07}{Initial version}
%
% \GetFileInfo{ustuttappendix.dtx}
%
% \DoNotIndex{\newcommand,\newenvironment}
% 
%
% \title{The \textsf{cskeleton} class\thanks{This document
%   corresponds to \textsf{cskeleton}~\fileversion, dated \filedate.}}
% \author{Scott Pakin \\ \texttt{scott+dtx@pakin.org}}
%
% \maketitle
%
% \section{Introduction}
%
% Put text here.
%
% \section{Usage}
%
% \StopEventually{\PrintChanges\PrintIndex}
%
% \section{Implementation}
%
% The package (once part of the exsheets package), provides a framework for
% providing multilingual features to a LaTeX package. The packge has its own
% basic dictionaries for English, Dutch, French, German and Spanish; it aims to
% use translation material for English, Dutch, French, German, Italian, Spanish,
% Catalan, Turkish, Croatian, Hungarian, Danish and Portuguese from babel or
% polyglossia if either is in use in the document. (Additional languages from
% the multilingual packages may be possible: ask the author.)
%    \begin{macrocode}
\RequirePackage{translations}
%    \end{macrocode}
%
% The |translator| package is a \LaTeX package that provides a flexible
% mechanism for translating individual words into different languages. For
% example, it can be used to translate a word like ``figure'' into, say, the
% German word ``Abbildung''. Such a translation mechanism is useful when the
% author of some package would like to localize the package such that texts are
% correctly translated into the language preferred by the user. The |translator|
% package is \textit{not} intended to be used to automatically translate more
% than a few words.
%    \begin{macrocode}
\RequirePackage{translator}
%    \end{macrocode}
%
% The appendix package provides various ways of formatting the titles of
% appendices. Also (sub)appendices environments are provided that can be used,
% for example, for per chapter/section appendices. The word 'Appendix' or
% similar can be prepended to the appendix number for article class documents.
% The word 'Appendices' or similar can be added to the table of contents before
% the appendices are listed. The word 'Appendices' or similar can be typeset as
% a |\part|-like heading (page) in the body. An appendices environment is
% provided which can be used instead of the |\appendix| command.
%    \begin{macrocode}
\PassOptionsToPackage{%
% add appendix to TOX
    toc,%
% Make a |\part| of |\chapter| page for the appendix starting page
    page,%
}{appendix}%
\RequirePackage{appendix}
%    \end{macrocode}
%
% We will make the appendix have a blue bar spanning the width of the page
% \begin{macro}{\appendixpage}
%    \begin{macrocode}
\renewcommand*{\appendixpage}{%
  \clear@ppage%
  \newgeometry{%
    inner=0cm,%
    outer=0cm,%
    top=0cm,%
    bottom=0cm,%
  }%
  \thispagestyle{plain}%
  \if@twocolumn\onecolumn\@tempswatrue\else\@tempswafalse\fi%
  \null\vfil%
  \markboth{}{}%
  {%
    \tikzexternaldisable%
    \begin{tikzpicture}[%
        remember picture,%
        overlay,%
        inner sep=0pt,%
        outer sep=0pt,%
        x=\paperwidth,%
        y=\paperheight,%
      ]%
      %
      \node[%
          draw=UStuttColorful,%
          fill=UStuttColorful,%
          inner sep=0pt,%
          outer sep=0pt,%
          align=center,%
          text width=1.05\paperwidth,%
          minimum height=4.00\baselineskip,%
          anchor=south,%
        ]%
        at (current page.center)%
        {%
          \interlinepenalty\@M{\usekomafont{part}{\color{UStuttWhite}{\appendixpagename}}}%
        };%
    \end{tikzpicture}%
    \tikzexternalenable%
  }%
  \if@dotoc@pp%
    \addappheadtotoc%
  \fi%
  \vfil\newpage%
  \if@twoside%
    \if@openright%
      \null%
      \thispagestyle{empty}%
      \newpage%
    \fi%
  \fi%
  \if@tempswa%
    \twocolumn%
  \fi%
  \restoregeometry%
}
%    \end{macrocode}
% \end{macro}
%
% \begin{macro}{\fixcleverefappendix}
% \cmd{\fixcleverefappendix}\\
% For some weird reason, if we want to use |cleverref| with the |appendix|
% package, it does not typeset the chapter prefix correctly. This changes this
% odd behavior.
%    \begin{macrocode}
\newcommand{\fixcleverefappendix}{%
  \crefalias{chapter}{appendix}
  \crefalias{section}{subappendix}
  \crefalias{subsection}{subsubappendix}
  \crefalias{subsubsection}{subsubsubappendix}
}
%    \end{macrocode}
% \end{macro}
%
% \subsection{Localization}
%
% Localization is implemented in a very fancy way. By making use of the
% |translator| package, we can offload any localized text to a |.dict| file
% while keeping the main |cls| file clean of any localization. In addition, we
% do not have to make sure we define each localizable string for every language
% and dialect that |babel| has to offer. Using |translator|, we define a
% |German| and |English| dictionary, that is automatically mapped to the
% respective languages loaded dialect(s). Cool, eh?
%    \begin{macrocode}
\usedictionary{ustutt}
%    \end{macrocode}
%
%    \begin{macrocode}
\renewcommand\appendixname{\translate{ustutt@appendix@name}}
\renewcommand\appendixtocname{\translate{ustutt@appendix@tocname}}
\renewcommand\appendixpagename{\translate{ustutt@appendix@pagename}}
%    \end{macrocode}
%
% \Finale
\endinput

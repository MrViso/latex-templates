% \iffalse meta-comment
%
% Copyright (C) 2019 by Philipp Tempel <latex@philipptempel.me>
% -------------------------------------------------------
% 
% This file may be distributed and/or modified under the
% conditions of the LaTeX Project Public License, either version 1.3
% of this license or (at your option) any later version.
% The latest version of this license is in:
%
%    http://www.latex-project.org/lppl.txt
%
% and version 1.3 or later is part of all distributions of LaTeX 
% version 2005/12/01 or later.
%
% \fi
%
% \iffalse
%<*driver>
\ProvidesFile{ustuttgloss.dtx}
%</driver>
%<package>\NeedsTeXFormat{LaTeX2e}[2005/12/01]
%<package>\ProvidesPackage{ustuttgloss}
%<*package>
    [2019/12/07 v1.0.3 University of Stuttgart glossaries file]
%</package>
%
%<*driver>
\documentclass{ltxdoc}
\EnableCrossrefs         
\CodelineIndex
\RecordChanges
\begin{document}
  \DocInput{ustuttgloss.dtx}
  \PrintChanges
  \PrintIndex
\end{document}
%</driver>
% \fi
%
% \CheckSum{0}
%
% \CharacterTable
%  {Upper-case    \A\B\C\D\E\F\G\H\I\J\K\L\M\N\O\P\Q\R\S\T\U\V\W\X\Y\Z
%   Lower-case    \a\b\c\d\e\f\g\h\i\j\k\l\m\n\o\p\q\r\s\t\u\v\w\x\y\z
%   Digits        \0\1\2\3\4\5\6\7\8\9
%   Exclamation   \!     Double quote  \"     Hash (number) \#
%   Dollar        \$     Percent       \%     Ampersand     \&
%   Acute accent  \'     Left paren    \(     Right paren   \)
%   Asterisk      \*     Plus          \+     Comma         \,
%   Minus         \-     Point         \.     Solidus       \/
%   Colon         \:     Semicolon     \;     Less than     \<
%   Equals        \=     Greater than  \>     Question mark \?
%   Commercial at \@     Left bracket  \[     Backslash     \\
%   Right bracket \]     Circumflex    \^     Underscore    \_
%   Grave accent  \`     Left brace    \{     Vertical bar  \|
%   Right brace   \}     Tilde         \~}
%
%
% \changes{v1.0}{2004/11/05}{Initial version}
%
% \GetFileInfo{ustuttgloss.dtx}
%
% \DoNotIndex{\newcommand,\newenvironment}
% 
%
% \title{The \textsf{ustuttgloss} package\thanks{This document
%   corresponds to \textsf{ustuttgloss}~\fileversion, dated \filedate.}}
% \author{Philipp Tempel \\ \texttt{latex@philipptempel.me}}
%
% \maketitle
%
% \section{Introduction}
%
% This package is intended if you want to have configuration-easy use of the
% glossaries package and just want a drop-in solution.
%
% \StopEventually{}
%
% \section{Implementation}
%
% \subsection{Load Packages}
%
%
% Load all the glossaries package and its useful extensions.
%    \begin{macrocode}
\PassOptionsToPackage{%
  automake,%
  acronyms,%
  toc,%
  xindy,%
  shortcuts,%
  nopostdot,%
  translate=babel,%
}{glossaries}
%    \end{macrocode}
%
% Glossaries extra: This package provides improvements and extra features to
% the glossaries package. Some of the default glossaries.sty behaviour is
% changed by glossaries-extra.sty. See the user manual
% glossaries-extra-manual.pdf for further details. glossaries-extra.sty
% requires the glossaries package and, naturally, all packages required by
% glossaries.sty. 
%    \begin{macrocode}
\PassOptionsToPackage{%
  automake,%
}{glossaries-extra}
%    \end{macrocode}
%
% Create glossaries and lists of acronyms. The glossaries package supports
% acronyms and multiple glossaries, and has provision for operation in several
% languages (using the facilities of either babel or polyglossia). New entries
% are defined to have a name and description (and optionally an associated
% symbol). Support for multiple languages is offered, and plural forms of terms
% may be specified. An additional package, glossaries-accsupp, can make use of
% the accsupp package mechanisms for accessibility support for PDF files
% containing glossaries. The user may define new glossary styles, and preambles
% and postambles can be specified. There is provision for loading a database of
% terms, but only terms used in the text will be added to the relevant glossary.
% The package uses an indexing program to provide the actual glossary; either
% makeindex or xindy may serve this purpose, and a Perl script is provided to
% serve as interface. The package distribution also provides the mfirstuc
% package, for changing the first letter of a word to upper case. The package
% supersedes the author's glossary package (which is now obsolete), and a
% conversion tool is provided.
%    \begin{macrocode}
\RequirePackage{glossaries}
\RequirePackage{glossaries-extra}
\RequirePackage{glossary-longbooktabs}
\RequirePackage{glossaries-extra-stylemods}
\setabbreviationstyle[acronym]{long-short}
\renewcommand{\glossarypreamble}{\glsfindwidesttoplevelname[\currentglossary]}
%    \end{macrocode}
%
% \begin{macro}{\glsaddkey}
% Define a new |unit| key for each glossary entry
%    \begin{macrocode}
\glsaddkey{unit}{%
  \glsentrytext{\glslabel}%
}{%
  \glsentryunit%
}{%
  \GLsentryunit%
}{%
  \glsunit%
}{%
  \Glsunit%
}{%
  \GLSunit%
}
%    \end{macrocode}
% \end{macro}
%
% 
% \begin{macro}{\glssetnoexpandfield}
% Do not expand anything inside the |unit| field so that we can typeset the unit
% with |\si|
%    \begin{macrocode}
\glssetnoexpandfield{unit}
%    \end{macrocode}
% \end{macro}
%
%
% \begin{macro}{\newglossarystyle}
% New glossary style for symbols
%    \begin{macrocode}
\newglossarystyle{isw-long-symbol}{%
  % Set the glossaries description to width of 0.65\linewidth
  \setlength\glsdescwidth{0.70\linewidth}%
  % Set the `theglossary` style
  \renewenvironment{theglossary}{% BEGIN
    \centering%
    \begin{longtable}{%
      % p{0.05\linewidth}% Symbol 
      p{\widthof{{Symbol}}}% Symbol
      p{\glsdescwidth}% Description
      p{\widthof{{Einheit}}}% Unit
    }%
  }{% END
    \end{longtable}%
  }%
  \renewcommand*{\glossaryheader}{%
    % \toprule 
    \bfseries \symbolname%
      & \bfseries \descriptionname%
      & \bfseries \unitname%
    \tabularnewline \midrule%
    \endhead%
  }%
  \renewcommand*{\glsgroupheading}[1]{%
    \multicolumn{3}{l}{}%
    \tabularnewline%
    % \midrule%
    \multicolumn{3}{l}{%
      \glstreegroupheaderfmt{\glsgetgrouptitle{##1}}%
      % \glsgetgrouptitle{##1}
      % \glstreegroupheaderfmt{%
      %   \glsnavhypertarget{##1}{\glsgetgrouptitle{##1}}%
      % }%
      % \indexspace%
    }%
    \tabularnewline \midrule%
  }%
  \renewcommand{\glossentry}[2]{%
    \glsentryitem{##1}\glstarget{##1}{\glossentryname{##1}}%
      & \glossentrydesc{##1}%
      & \glsunit{##1}%
      % & \ifglshasfield{##1}{unit}{\glsunit{##1}}{}%
    \tabularnewline%
  }%
  \renewcommand{\subglossentry}[3]{%
    \glssubentryitem{##2}\glstarget{##2}{\glossentryname{##2}}%
      & \glossentrydesc{##2}
      & \ifglshasfield{##1}{unit}{\glsunit{##1}}{}%
    \tabularnewline%
  }%
  \renewcommand*{\glsgroupskip}{%
    \ifglsnogroupskip \else & & \tabularnewline\fi%
  }%
}
%    \end{macrocode}
%
% Glossary style for list of symbols without grouping
%    \begin{macrocode}
\newglossarystyle{isw-long-symbol-nogroup}{%
  \setglossarystyle{isw-long-symbol}%
  \renewcommand*{\glsgroupheading}[1]{}%
  \renewcommand{\glsgroupskip}{}%
}
%    \end{macrocode}
%
%
% New glossary style for acronyms
%    \begin{macrocode}
\newglossarystyle{isw-long-acronym}{%
  % Set the glossaries description to width of 0.65\linewidth
  \setlength\glsdescwidth{0.64\linewidth}%
  % Set the `theglossary` style
  \renewenvironment{theglossary}{% BEGIN
    \centering%
    \begin{longtable}{%
      p{\widthof{{Abk\"urzung}}}% Acronym
      p{\glsdescwidth}% Description
    }%
  }{% END
    \end{longtable}%
  }%
  \renewcommand*{\glossaryheader}{%
    % \toprule 
    \bfseries \acronymname%
      & \bfseries \descriptionname%
    \tabularnewline \midrule%
    \endhead%
  }%
  \renewcommand*{\glsgroupheading}[1]{%
    \multicolumn{2}{l}{%
      \glstreegroupheaderfmt{\glsgetgrouptitle{##1}}%
    }%
    \tabularnewline \midrule%
  }%
  \renewcommand{\glossentry}[2]{%
    \glsentryitem{##1}\glstarget{##1}{\glossentryname{##1}}%
      & \glossentrydesc{##1}%
    \tabularnewline%
  }%
  \renewcommand{\subglossentry}[3]{%
    \glssubentryitem{##2}\glstarget{##2}{\glossentryname{##2}}%
      & \glossentrydesc{##2}
    \tabularnewline%
  }%
  \renewcommand*{\glsgroupskip}{%
    \ifglsnogroupskip \else & \tabularnewline\fi%
  }%
}
%    \end{macrocode}
%
%
% Glossary style for acronyms without group heading
%    \begin{macrocode}
\newglossarystyle{isw-long-acronym-nogroup}{%
  \setglossarystyle{isw-long-acronym}%
  \renewcommand*{\glsgroupheading}[1]{}%
  \renewcommand{\glsgroupskip}{%
    \ifglsnogroupskip \else & \tabularnewline\fi%
  }%
}
%    \end{macrocode}
%
%
% New glossary style for notation
%    \begin{macrocode}
\newglossarystyle{isw-long-notation}{%
  % Set the glossaries description to width of 0.65\linewidth
  \setlength\glsdescwidth{0.74\linewidth}%
  % Set the `theglossary` style
  \renewenvironment{theglossary}{% BEGIN
    \centering%
    \begin{longtable}{%
      p{\widthof{{Eintrag}}}% Notation
      p{\glsdescwidth}% Description
    }%
  }{% END
    \end{longtable}%
  }%
  \renewcommand*{\glossaryheader}{%
    % \toprule 
    \bfseries \symbolname%
    & \bfseries \descriptionname%
    \tabularnewline \midrule%
    \endhead%
  }%
  \renewcommand*{\glsgroupheading}[1]{%
    \multicolumn{2}{l}{}%
    \tabularnewline%
    % \midrule%
    \multicolumn{2}{l}{%
      \glstreegroupheaderfmt{\glsgetgrouptitle{##1}}%
    }%
    \tabularnewline \midrule%
  }%
  \renewcommand{\glossentry}[2]{%
    \glsentryitem{##1}\glstarget{##1}{\glossentrysymbol{##1}}%
      & \glossentrydesc{##1}%
    \tabularnewline%
  }%
  \renewcommand{\subglossentry}[3]{%
    \glssubentryitem{##2}\glstarget{##2}{\glossentrysymbol{##2}}%
      & \glossentrydesc{##2}
    \tabularnewline%
  }%
  % \renewcommand*{\glsgroupskip}{}
  \renewcommand*{\glsgroupskip}{%
    \ifglsnogroupskip \else & \tabularnewline\fi
  }%
}
%    \end{macrocode}
%
%
% Glossary style for list of symbols without grouping
%    \begin{macrocode}
\newglossarystyle{isw-long-notation-nogroup}{%
  \setglossarystyle{isw-long-notation}%
  \renewcommand*{\glsgroupheading}[1]{}%
  \renewcommand{\glsgroupskip}{}%
}
%    \end{macrocode}
% \end{macro}
%
%
% \Finale
\endinput
